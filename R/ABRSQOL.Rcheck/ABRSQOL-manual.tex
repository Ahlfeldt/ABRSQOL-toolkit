\nonstopmode{}
\documentclass[a4paper]{book}
\usepackage[times,inconsolata,hyper]{Rd}
\usepackage{makeidx}
\makeatletter\@ifl@t@r\fmtversion{2018/04/01}{}{\usepackage[utf8]{inputenc}}\makeatother
% \usepackage{graphicx} % @USE GRAPHICX@
\makeindex{}
\begin{document}
\chapter*{}
\begin{center}
{\textbf{\huge Package `ABRSQOL'}}
\par\bigskip{\large \today}
\end{center}
\ifthenelse{\boolean{Rd@use@hyper}}{\hypersetup{pdftitle = {ABRSQOL: Quality-of-life solver for "Measuring quality of life under spatial frictions"}}}{}
\begin{description}
\raggedright{}
\item[Type]\AsIs{Package}
\item[Title]\AsIs{Quality-of-life solver for ``Measuring quality of life under
spatial frictions''}
\item[Version]\AsIs{1.0.0}
\item[Author]\AsIs{Gabriel M Ahlfeldt }\email{g.ahlfeldt@hu-berlin.de}\AsIs{}
\item[Maintainer]\AsIs{Max von Mylius }\email{max.mylius@hu-berlin.de}\AsIs{}
\item[Description]\AsIs{This toolkit implements a numerical solution algorithm 
to invert a quality of life (QoL) from observed data in various 
programming languages. The QoL measure is based on Ahlfeldt, 
Bald, Roth, Seidel (2024); Measuring quality of life under 
spatial frictions. Unlike the traditional Rosen-Roback measure,
this measure accounts for mobility frictions—generated by 
idiosyncratic tastes and local ties—and trade frictions—generated
by trade costs and non-tradable services, thereby reducing
non-classical measurement error. When using this programme or
the toolkit in your work, please cite the paper.}
\item[URL]\AsIs{}\url{https://github.com/Ahlfeldt/ABRSQOL-toolkit#readme}\AsIs{}
\item[BugReports]\AsIs{}\url{https://github.com/Ahlfeldt/ABRSQOL-toolkit/issues}\AsIs{}
\item[License]\AsIs{MIT + file LICENSE}
\item[Encoding]\AsIs{UTF-8}
\item[LazyData]\AsIs{false}
\item[Depends]\AsIs{R (>= 2.10)}
\item[RoxygenNote]\AsIs{7.3.2}
\item[Suggests]\AsIs{testthat (>= 3.0.0)}
\item[Config/testthat/edition]\AsIs{3}
\end{description}
\Rdcontents{Contents}
\HeaderA{ABRSQOL}{ABRSQOL numerical solution algorithm to invert a quality of life measure}{ABRSQOL}
%
\begin{Description}
This toolkit implements a numerical solution algorithm
to invert a quality of life (QoL) from observed data
in various programming languages. The QoL measure is
based on Ahlfeldt, Bald, Roth, Seidel (2024):
Measuring quality of life under spatial frictions.
Unlike the traditional Rosen-Roback measure, this measure
accounts for mobility frictions—generated by idiosyncratic
tastes and local ties—and trade frictions—generated by
trade costs and non-tradable services, thereby reducing
non-classical measurement error.
When using this programme or the toolkit in your work, please cite the paper.
\end{Description}
%
\begin{Usage}
\begin{verbatim}
ABRSQOL(
  df,
  w = "w",
  p_H = "p_H",
  P_t = "P_t",
  p_n = "p_n",
  L = "L",
  L_b = "L_b",
  alpha = 0.7,
  beta = 0.5,
  gamma = 3,
  xi = 5.5,
  conv = 0.5,
  tolerance = 1e-10,
  maxiter = 10000
)
\end{verbatim}
\end{Usage}
%
\begin{Arguments}
\begin{ldescription}
\item[\code{df}] input data containing variables refenced by 
following arguments: data.frame or matrix

\item[\code{w}] wage index variable name or column index:
character or integer (or list), default='w'

\item[\code{p\_H}] floor\_space\_price variable name or column index:
character or integer, default='p\_H'

\item[\code{P\_t}] tradable\_goods\_price variable name or column index:
character or integer, default='P\_t'

\item[\code{p\_n}] local\_services\_price variable name or column index:
character or integer, default='p\_n'

\item[\code{L}] residence\_population variable name or column index:
character or integer, default='L'

\item[\code{L\_b}] hometown\_population variable name or column index:
character or integer, default='L\_b'

\item[\code{alpha}] Income share on non-housing consumtpion:
double, default=0.7

\item[\code{beta}] Share of tradable goods in non-housing consumption:
double, default=0.5

\item[\code{gamma}] Idiosyncratic taste dispersion (inverse labour
supply elasticity):
double, default=3

\item[\code{xi}] Valuation of local ties: double, default=5

\item[\code{conv}] Convergence parameter (Hgher value increases spead of,
convergence and risk of bouncing): double, default=0.5

\item[\code{tolerance}] Value used in stopping rule (The mean absolute error (MAE).
Smaller values imply greater precision and longer convergence):
double, default=1e-10

\item[\code{maxiter}] Maximum number of iterations after which the algorithm
is forced to stop: integer, default=1e4
\end{ldescription}
\end{Arguments}
%
\begin{Details}
Notice that quality of life is identified up to a constant.
Therefore, the inverted QoL measures measure has a relative
interpretation only. We normalize the QoL relative to the first
observation in the data set. It is straightforward to rescale
the QoL measure to any other location or any other value (such
as the mean or median in the distribution of QoL across locations).
\end{Details}
%
\begin{Value}
inverted quality of life measure as Numeric vector
(identified up to a constant)
\end{Value}
%
\begin{Examples}
\begin{ExampleCode}
# Example 1: load testdata, run QoL inversion with default parameters, append and view result
data(ABRSQOL_testdata)
my_dataframe <- ABRSQOL_testdata
my_dataframe$qol1 <- ABRSQOL(df=ABRSQOL_testdata)
View(my_dataframe)

# Example 2: load your data from csv, run inversion, save result as csv
# my_dataframe <- read.csv("path/to/your/csv_filename.csv")
# my_dataframe$qol2 <- ABRSQOL(
#  # supply your dataset as a dataframe
#  df=my_dataframe,
#  # and specify the corresponding variable name for your dataset
#  w = 'wage',
#  p_H = 'floor_space_price',
#  P_t = 'tradable_goods_price',
#  p_n = 'local_services_price',
#  L = 'residence_pop',
#  L_b = 'L_b',
#  # freely adjust remaining parameters
#  alpha = 0.7,
#  beta = 0.5,
#  gamma = 3,
#  xi = 5.5,
#  conv = 0.3,
#  tolerance = 1e-11,
#  maxiter = 50000
#)
#write.csv(my_dataframe, 'qol_of_my_data.csv'

# Example 3: Reference variables in your dataset by using the column index
my_dataframe$qol3 <- ABRSQOL(
  df=my_dataframe,
  w = 1,
  p_H = 3,
  P_t = 4,
  p_n = 2,
  L = 6,
  L_b = 5
)

# Example 4: Having named the variables in your data according to the default parameters, you can ommit specifying variable names
my_dataframe$qol4 <- ABRSQOL(
  df=my_dataframe,
  alpha = 0.7,
  beta = 0.5,
  gamma = 3,
  xi = 5.5,
  conv = 0.5
)

\end{ExampleCode}
\end{Examples}
\HeaderA{ABRSQOL\_testdata}{Test data for ABRSQOL quality of life inversion}{ABRSQOL.Rul.testdata}
\keyword{datasets}{ABRSQOL\_testdata}
%
\begin{Description}
This is a test data set, and it is not identical to the data used in the
paper. The data set includes average disposable household income as a
measure of wage, the local labour market house price index from Ahlfeldt,
Heblich, Seidel (2023), the 2015 census population as a measure of
residence population and hte 1985 census population as measure of hometown
population. Tradable goods price and local services price indices
are uniformly set to one.
\end{Description}
%
\begin{Usage}
\begin{verbatim}
data(ABRSQOL_testdata)
\end{verbatim}
\end{Usage}
%
\begin{Format}
data.frame with colnames=c('llm\_id','w','p\_H','P\_t','p\_n','L','L\_b',
'Name','coord\_x','coord\_y') and 141 observations
\end{Format}
%
\begin{Source}
\Rhref{https://https://github.com/Ahlfeldt/ABRSQOL-toolkit}{ABRSQOL-toolkit}
\end{Source}
%
\begin{References}
Gabriel M. Ahlfeldt, Fabian Bald, Duncan Roth, Tobias Seidel
(forthcoming): Measuring quality of life under spatial frictions.
\end{References}
%
\begin{Examples}
\begin{ExampleCode}
library('ABRSQOL')
data(ABRSQOL_testdata)
ABRSQOL_testdata$QoL = ABRSQOL(df=ABRSQOL_testdata)
View(ABRSQOL_testdata)
\end{ExampleCode}
\end{Examples}
\printindex{}
\end{document}
